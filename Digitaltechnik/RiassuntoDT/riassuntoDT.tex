\documentclass[a4paper]{article}

\usepackage{amsmath}
\usepackage{tabularx}

\begin{document}

%\title{Title}
%\author{Author}
%\date{Today}
%\maketitle



\section{Organization and Introduction}
\begin{itemize}
\item The Art of managing complexity
\begin{itemize}
\item Abstraction: Hiding details when they are not important
\item Discipline: Intentionally restricting your design choices to that you can work more productively at higher abstraction levels
\item The three -Y's
\begin{itemize}
\item Hierarchy: A system is divided into modules of smaller complexity
\item Modularity: Having well defined functions and interfaces
\item Regularity: Encouraging uniformity, so modules can be easily re-used
\end{itemize}
\end{itemize}
\item Bit: \textbf{B}inary dig\textbf{it}
\end{itemize}

\section{Binary Numbers}
\begin{itemize}
\item Powers of two:\\
\begin{tabular}{c|c|c}
$2^0= 1$ & $2^5=32$ & $2^{10}= 1024$\\ 
$2^1= 2$ & $2^6=64$ & $2^{11}= 2048$\\ 
$2^2= 4$ & $2^7=128$ & $2^{12}= 4096$\\ 
$2^3= 8$ & $2^8=256$ & $2^{13}= 8192$\\ 
$2^4= 16$ & $2^9=512$ & $2^{14}= 16384$\\ 
\end{tabular}
\item Binary to decimal conversion
\begin{align*}
10011_2&=2^4\times 1 +2^3\times 0 + 2^2\times 0 +2^1\times 1 +2^0\times 1\\
&=16 \times 1+8 \times 0+4 \times 0+2 \times 1+1 \times 1\\
&=16+0+0+2+1=19_{10}
\end{align*}
\item Convert decimal to binary (roughly). Example with $47_{10}$ to binary\\
\begin{tabular}{c|c|c|c|c}
$2^6=64$& is $64\leq 47$?&no&0&do nothing\\ 
$2^5=32$& is $32\leq 47$?&yes&1&47-32=15\\
$2^4=16$& is $16\leq 15$?&no&0&do nothing\\
$2^3=8$& is $8\leq 15$?&yes&1&15-8=7\\ 
$2^2=4$& is $4\leq 7$?&yes&1&7-4=3\\
$2^1=2$& is $2\leq 3$?&yes&1&3-2=1\\
$2^0=1$& is $1\leq 1$?&yes&1&1-1=0; done!\\
\end{tabular}\\
$\Rightarrow 47_{10}$ to binary is $0101111_2$ 
\item Binary values and range\\
\begin{tabularx}{15cm}{X|X|X|X|}
{}& How many values? & Range? & Example: 3-digit number\\\hline
$N-$digit decimal number & $10^N$ & $\lbrack 0,10^{N}-1\rbrack$ & $10^3=1000$ possible values, range:$\lbrack 0,999\rbrack$\\
$N-$bit binary number & $2^N$ & $\lbrack 0,2^{N}-1\rbrack$ & $2^3=8$ possible values, range:$\lbrack 0,7\rbrack=\lbrack 000_2\text{ to }111_2\rbrack$\\
\end{tabularx}
\item Hexadecimal (Base-16) Numbers\\
\begin{tabular}{c|c|c|c|c|c|c}
Decimal & Hexadecimal & Binary&{}&Decimal & Hexadecimal & Binary\\\hline
0&0&0000&{}&8&8&1000\\
1&1&0001&{}&9&9&1001\\
2&2&0010&{}&10&A&1010\\
3&3&0011&{}&11&B&1011\\
4&4&0100&{}&12&C&1100\\
5&5&0101&{}&13&D&1101\\
6&6&0110&{}&14&E&1110\\
7&7&0111&{}&15&F&1111\\
\end{tabular}
\item Bits, Bytes, Nibbles...
\[\begin{array}{*{20}{c}}
{\underbrace 1_{{\text{MSB}}}001011\underbrace 0_{{\text{LSB}}}}&{\overbrace {1001\underbrace {0110}_{{\text{nibble}}}}^{{\text{Byte}}}}&{\underbrace {{\text{CE}}}_{{\text{MSB}}}{\text{BF9A}}\underbrace {{\text{D7}}}_{{\text{LSB}}}}
\end{array}\]
Where MSB=Most significant Bit and LSB=Least significant Bit
\item Addition in base two works exactly the same as in base 10, using carries
\item Overflow
\begin{itemize}
\item Digital systems operate on a fixed number of bits
\item Addition overflows when the result is too big to fit in the available number of bits
\end{itemize}
\item Signed Binary Numbers
\begin{itemize}
\item Sign/Magnitude Numbers
\begin{itemize}
\item 1 sign bit, $N-1$ magnitude bits
\item Sign bit is the most significant (left-most) bit
\item Example: 4-bit sign/mag repr. of $\pm 6$:
\begin{itemize}
\item $+6=\textbf{0}110$
\item $-6=\textbf{1}110$
\end{itemize}
\item Range of an $N-$bit sign/magnitude number:\\
$\lbrack -\left( 2^{N-1}-1\right),2^{N-1}-1\rbrack$
\item Problems:
\begin{itemize}
\item Addition doesn't work
\item Two representations of 0 ($\pm 0$): 1000 and 0000
\item Introduces complexity in the processor design
\end{itemize}
\end{itemize}
\item One's Complement Numbers
\begin{itemize}
\item A negative number is formed by reversing the bits of the positive number (MSB still indicates the sign of the integer)\\
\begin{tabular}{|c|c|c|c|c|c|c|c|c|c|c|}
\hline
$2^7$&$2^6$&$2^5$&$2^4$&$2^3$&$2^2$&$2^1$&$2^0$&{}&One's Compl.&Unsigned\\\hline\hline
0&0&0&0&0&0&0&0&$=$&0&0\\
0&0&0&0&0&0&0&1&$=$&1&1\\
0&0&0&0&0&0&1&0&$=$&2&2\\
\dots&\dots&\dots&\dots&\dots&\dots&\dots&\dots&\dots&\dots&\dots\\
0&1&1&1&1&1&1&1&$=$&127&127\\
1&0&0&0&0&0&0&0&$=$&-127&128\\
1&0&0&0&0&0&0&1&$=$&-126&129\\
\dots&\dots&\dots&\dots&\dots&\dots&\dots&\dots&\dots&\dots&\dots\\
1&1&1&1&1&1&0&1&$=$&-2&253\\
1&1&1&1&1&1&1&0&$=$&-1&254\\
1&1&1&1&1&1&1&1&$=$&-0&255\\\hline
\end{tabular}
\item Range of $n-$bit number: $\lbrack -2^{n-1}-1,2^{n-1}-1\rbrack$, 8 bits:$\lbrack -127,127\rbrack$
\item Addition: Done using binary addition with end-around carry. If there is a carry out of the MSB of the sum, this bit must be added to the LSB of the sum
\end{itemize}
\item Two's Complement Numbers
\begin{itemize}
\item Don't have same problems as sign/magnitude numbers:
\begin{itemize}
\item addition works
\item Single representation for 0
\end{itemize}
\item Has advantages over one's complement:
\begin{itemize}
\item Has a single 0 representation
\item Eliminates the end-around carry operation required in one's complement addition.
\end{itemize}

\item A negative number is formed by reversing the bits of the positive number (MSB still indicates the sign of the integer) and adding 1:\\
\begin{tabular}{|c|c|c|c|c|c|c|c|c|c|c|}
\hline
$2^7$&$2^6$&$2^5$&$2^4$&$2^3$&$2^2$&$2^1$&$2^0$&{}&Two's Compl.&Unsigned\\\hline\hline
0&0&0&0&0&0&0&0&$=$&0&0\\
0&0&0&0&0&0&0&1&$=$&1&1\\
0&0&0&0&0&0&1&0&$=$&2&2\\
\dots&\dots&\dots&\dots&\dots&\dots&\dots&\dots&\dots&\dots&\dots\\
0&1&1&1&1&1&1&1&$=$&127&127\\
1&0&0&0&0&0&0&0&$=$&-128&128\\
1&0&0&0&0&0&0&1&$=$&-127&129\\
\dots&\dots&\dots&\dots&\dots&\dots&\dots&\dots&\dots&\dots&\dots\\
1&1&1&1&1&1&0&1&$=$&-3&253\\
1&1&1&1&1&1&1&0&$=$&-2&254\\
1&1&1&1&1&1&1&1&$=$&-1&255\\\hline
\end{tabular}
\item Same as unsigned binary, but the most significant bit (MSB) has value of $-2^{N-1}$
\begin{itemize}
\item Most positive 4-bit number: 0111
\item Most negative 4-bit number: 1000
\end{itemize}
\item The most significant bit still indicates the sign (1=neg., 0=pos.)
\item Range of an $N-$bit two's comp. number: $\lbrack -2^{N-1},2^{N-1}-1\rbrack$, 8 bits:$\lbrack -128,127\rbrack$
\end{itemize}
\end{itemize}
\item Increasing bit width (assume from $N$ to $M$, with $M>N$): 
\begin{itemize}
\item Sign-extension
\begin{itemize}
\item Sign bit is copied into MSB
\item Number value remains the same
\item Give correct result for two's compl. numbers
\item Example 1:
\begin{itemize}
\item 4-bit representation of $3=\textbf{0}011$
\item 8-bit sign-extended value: $\textbf{00000}011$
\end{itemize}
\item Example 2:
\begin{itemize}
\item 4-bit representation of $-5=\textbf{1}011$
\item 8-bit sign-extended value: $\textbf{11111}011$
\end{itemize}
\end{itemize}
\item Zero-extension
\begin{itemize}
\item Zeros are copied into MSB
\item Value will change for negative numbers
\item Example 1:
\begin{itemize}
\item 4-bit value: $0011_2=3_{10}$
\item 8-bit zero-extended value: $\textbf{0000}0011_2=3_{10}$
\end{itemize}
\item Example 2:
\begin{itemize}
\item 4-bit value: $1011_2=-5_{10}$
\item 8-bit zero-extended value: $\textbf{0000}1011_2=\textbf{$11_{\textbf{10}}$}$
\end{itemize}
\end{itemize}
\end{itemize}
\end{itemize}

\section{Short Introduction to Electrical Engineering (EE Perspective)}
\begin{itemize}
\item The goal of circuit design is to optimize:
\begin{itemize}
\item Area: Net circuit area is proportional to the cost of the device
\item Speed/Throughput: We want circuits that work  faster, or do more
\item Power/Energy
\begin{itemize}
\item Mobile devices need to work with a limited power supply
\item High performance devices dissipate more than 100W/$cm^2$
\end{itemize}
\item Design time
\begin{itemize}
\item Designers are expensive
\item The competition will not wait for you
\end{itemize}
\end{itemize}
\item (Frank's) Principles for engineering
\begin{itemize}
\item Good engineers are lazy: They do not want to work unnecessarily, be creative
\item They know how to ask the question ``why''?: take nothing for granted
\item Engineering is not a religion: Use what works best for you
\item Keep it simple and stupid: Engineers' job is to manage complexity
\end{itemize}
%Page 13 of slides
\end{itemize}
\end{document}