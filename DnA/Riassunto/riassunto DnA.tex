\documentclass[a4paper]{book}
\usepackage[utf8]{inputenc}
\usepackage[italian]{babel}
\usepackage{amsmath}
\usepackage{amsfonts}
\usepackage{amssymb}
\usepackage{color}
\usepackage{listings}
\definecolor{dkgreen}{rgb}{0,0.6,0}
\definecolor{gray}{rgb}{0.5,0.5,0.5}
\definecolor{mauve}{rgb}{0.58,0,0.82}
\lstset{frame=tb,
  language=Java,
  aboveskip=3mm,
  belowskip=3mm,
  showstringspaces=false,
  columns=flexible,
  basicstyle={\small\ttfamily},
  numbers=none,
  numberstyle=\tiny\color{gray},
  keywordstyle=\color{blue},
  commentstyle=\color{dkgreen},
  stringstyle=\color{mauve},
  breaklines=true,
  breakatwhitespace=true
  tabsize=3
}
\author{Sandro Marcon}
\title{Riassunto DnA}
\begin{document}
\chapter{Introduzione}
\section{Notazione O-$\Omega$-$\Theta$ (pag 4)}
Per semplificare l'analisi asintotica degli algoritmi sono state introdotte le 
seguenti notazioni:
$$O(f)=\{ g|\exists a>0: \exists b>0: \forall N \in N : g(N)\leq af(N)+b \}$$
$$\Omega (f)=\{ g|\exists c>0: \exists \mbox{ infinite n } : g(n) \geq cf(n) \}$$
In entrambi i casi si tratta di classi di funzioni. Quando $f \in O(g)$ e $f \in \Omega (g)$ al contempo si dice che $f \in \Theta (g)$. Esistono ulteriori definizioni alternative, ad esempio quelle utilizzate nel Blatt 1. Il seguente teorema può essere utile (dimostrazione nel Blatt 1):
\newtheorem{theorem}{Theorem}
\begin{theorem}
Date due funzioni $ f,g: N \rightarrow R^{+} $. Se $ \lim_{x \rightarrow +\infty} \frac{f(n)}{g(n)} $ converge ad una costante $C\geq 0$ allora $f \in O(g)$.
\end{theorem}
\section{Algoritmo di Karatsuba (pag 12)}
Un esempio di algoritmo più efficiente per multiplicare due numeri è il seguente:
$$65 * 28 = (2 * 6) * 100 + (2 * 6) * 10 + (5 * 8) * 10 + 5 * 8 + (6-5)*(8-2) * 10 = 1820 $$
In questo modo abbiamo solo 3 multiplicazioni elementari anziché 4. L'algoritmo può essere generalizzato grazie a divide and conquer dividendo ogni numero in due ed applicando l'algoritmo di base. Analizzando il tempo in base al numero delle multimplicazioni otteniamo che impieda circa $O(n^{1,58})$.
\section{Maximum subarray (pag 20)}
Dato un array di numeri il problema consiste nella ricerca del subarray con la somma degli elementi maggiore. Se essa è negativa il risultato è 0. Il metodo più efficiente per ricavare il risultato è il seguente:
\begin{lstlisting}
//A=array da 1, ... n	
max=0	
scan=0
for (i=1; i<=n; i++){	
	scan+=A[i]
	if scan $<$ 0
		scan=0
	if scan $>$ max 
		max=scan
}			
\end{lstlisting}
In questo modo il problema viene risolto in tempo lineare.	
\chapter{Sort}
Per semplicità ammettiamo che si debba sempre ordinare un array (chiamato a) contenente n numeri (interi). In ogni caso con questi algoritmi è possibile ordinare qualsiasi oggetto appartenente ad un universo in cui vige un ordine totale.
\section{Sortieren durch Auswahl (pag 82)}
Selection sort consiste nel cercare ogni volta il minimo tra la posizione i e n. Una volta trovato esso viene scambiato con l'i-tesimo numero. 
\subsubsection*{Esempio}
\[\begin{array}{*{20}{c}}
{15}&2&{43}&{17}&4\\
2&{15}&{43}&{17}&4\\
2&4&{43}&{17}&{15}\\
2&4&{15}&{17}&{43}
\end{array}\]
\subsubsection*{Implementazione}
\begin{lstlisting}
int i,j, min, temp
for i=1:n-1
	min=i
	for j=(i+1):n
		if a[j]$<$ a[min]
			min=j
	temp=a[min]
	a[min]=a[i]
	a[i]=temp
\end{lstlisting}
\subsubsection*{Analisi}

Dal doppio loop si vede semplicemente che l'algoritmo impiega $\Theta (n^2)$ comparazioni e nel peggior caso (i numeri sono ordinati dal più grande al più piccolo) O(n) scambi. Da notare che per trovare il minimo sono necessari almeno n-1 confronti (Satz 2.1), quindi l'algoritmo non può andare più veloce di così. 
\section{Sortieren durch Einfügen (pag 85)}
In inglese si chiama insertion sort. Per induzione i numeri fino a i-1 sono già ordinati. Il principio consiste di piazzare l'i-tesimo elemento al giusto posto, se necessario scalando i restanti a destra di una posizione. Esempio:
\[\begin{array}{*{20}{c}}
2&{15}&/&{43}&{17}&4\\
2&{15}&{43}&/&{17}&4\\
2&{15}&{17}&{43}&/&4\\
2&4&{15}&{17}&{43}&/
\end{array}\]
\subsubsection*{Implementazione}
\begin{lstlisting}
int i,j,t
for i=2:n
	j=i
	t=a[i]
	while a[j-1] > t
		//scala a destra di una posizione
		a[j]=a[j-1]
		j=j-1
	a[j]=t
\end{lstlisting}
Si nota subito che se implementato così l'algoritmo può non fermarsi se t è il più piccolo numero. Serve quindi un elemento di stop, ad esempio inserendo a[0]=t prima del while loop.

\subsubsection*{Analisi}

Nel peggior caso $\Theta (n^2)$ comparazioni ed altrettanti spostamenti. Nel miglior caso $O(n^2)$ comparazioni e spostamenti. Il caso medio rispecchia il peggiore.

\section{Bubblesort}
Il principio di questo algoritmo è semplicissimo: ad ogni iterazione viene scambiato l'elemento a[i] con a[i+1] (chiaramente solo se maggiore). In questo modo l'elemento più grande si sposta verso destra. Esempio:
\[\begin{array}{*{20}{c}}
{15}&2&{43}&{17}&4\\
2&{15}&{43}&{17}&4\\
2&{15}&{17}&{43}&4\\
2&{15}&{17}&4&{43}\\
2&{15}&4&{17}&{43}\\
2&4&{15}&{17}&{43}
\end{array}\]
\subsubsection*{Implementazione}
\begin{lstlisting}
do
	flag=true
	for i=1:n-1
	if a[i] $>$ a[i+1] 
		swap(a[i],a[i+1])
		flag=false
while (!flag)
\end{lstlisting}
\subsubsection*{Analisi}

Nel miglior caso, se l'array è già ordinato, abbiamo n-1 paragoni e nessuno scambio. Nel caso medio e peggiore l'algoritmo necessita di $\Theta (n^2)$ scambi e paragoni.


\end{document}