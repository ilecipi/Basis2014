\documentclass[a4paper]{article}
\usepackage{todonotes}
\usepackage{amsmath}

\usepackage{listings}
\usepackage{color}
\usepackage{enumerate}

%Colors for the listings
\definecolor{dkgreen}{rgb}{0,0.6,0}
\definecolor{gray}{rgb}{0.5,0.5,0.5}
\definecolor{mauve}{rgb}{0.58,0,0.82}
%style setup for the listings
\lstset{frame=tb,
  language=Eiffel,
  aboveskip=3mm,
  belowskip=3mm,
  showstringspaces=false,
  columns=flexible,
  basicstyle={\small\ttfamily},
  numbers=none,
  numberstyle=\tiny\color{gray},
  keywordstyle=\color{blue},
  commentstyle=\color{dkgreen},
  stringstyle=\color{blue},
  breaklines=true,
  breakatwhitespace=true
  tabsize=3
}

\newcommand{\inline}[1]{\lstinline!#1!}%Inline code


\begin{document}

\section{Introduction}
\subsection{Basic definitions}
\begin{itemize}
\item Feature
\begin{itemize}
\item Command: No return value, but does modify objects. On the synctactical level, it is an instruction
\item Query: Returns a value, but does not modify any objects. The syntax equivalent is the expression.
\begin{itemize}
\item Functions get their results through computation
\item Attributes are values directly stored in memory
\end{itemize}
For queries, there is the uniform access principle, which states that it doesn't matter to the client whether a query is implemented as a function or attribute. Features should be accessible to clients the same way whether implemented by storage or by computation.
\item Creation Procedure: Commands to initiate objects, can be several. There is also a \textsl{default\_create}, which is inherited by all classes, and does nothing by default. 
\end{itemize}
\item Feature Calls
\begin{itemize}
\item Unqualified calls: Feature calls which apply to the current object
\item Qualified calls: Feature calls which apply to a certain object, causing this object to become the current object. 
\end{itemize}
\item Class clauses
\begin{itemize}
\item Indexing
\item Inheritance
\item Creation
\item Feature
\item Invariant
\end{itemize}
\item Specimen: A syntactic element, such as a class name or an instruction, but no delimiters. The type of a specimen is its construct. See Describing syntax\todo{Add reference}
\item Abstract syntax tree: Shows the syntax structure with all its specimens, but obviously without any delimiters, A tree has nodes, each one of the following kind:
\begin{itemize}
\item Root: Node with no incoming branch.
\item Leaf: Node without outgoing branches
\item Internal node: Neither of the former
\end{itemize}
\item Basic elements of a program text:
\begin{itemize}
\item Terminals
\begin{itemize}
\item Identifiers: Names chosen by the programmer
\item Keywords
\item Special symbols, such as a period
\end{itemize}
\end{itemize}
\item Describing a program
\begin{itemize}
\item Semantic rules: Define the effect of programming, satisfying the syntax rules
\item Syntax rules: Define how to make up specimens out of tokens satisfying the lexical rules
\item Lexical rules: Define how to make up tokens out of characters
\end{itemize}
\item Syntax: The way you write a program; characters grouped into words, grouped into bigger structures. 
\item Semantics: The effect you expect from this program at runtime
\item Identifier: Name chosen by the programmer to represent certain program elements, such as classes, features or runtime values. If it denotes a runtime value, it is called an identity or variable if it can change its value. During execution, an entity may become attached to an object. 
\item Executing a system consists of creating a root object, which in an instance of a designated class from the system, the root class, using a designated creation procedure, called its root procedure.
\end{itemize}
\subsection{Variables}
\begin{itemize}
\item Types
\begin{itemize}
\item Reference types: Entities with a reference value
\item Expanded types: Entities with an object as a value
\item A type is one of:
\begin{itemize}
\item A non-generic class
\item A generic derivation, i.e. the name of a class followed by a list of types, the actual generic parameters, in brackets
\end{itemize}
\end{itemize}
\item Setters: It is possible to make assignments such as $x.att:=val$, which is shorthand for $x.set\_att(val)$
\item Effect of an assignment
\begin{itemize}
\item Reference types: Reference assignment
\item Expanded types: Value copy
\end{itemize}
\item Variable copy
\begin{itemize}
\item Shallow object duplication (creates a new object): $b:=a.twin$
\item Deep object duplication (creates a new object): $b:=a.deep\_twin$
\item Shallow field-by-field copy (does not create an object): $b.copy(a)$
\end{itemize}
\end{itemize}
\todo[inline]{Maybe add the objet creation diagram??}

\subsection{Interface}
\begin{itemize}
\item A client of a software mechanism is a system of any kind - such as a software element or a human - that uses it. For its client, the mechanism is a supplier
\item Interface: The description of techniques enabling clients to use these mechanisms. For example: GUIs (Graphical User Interface), command line interfaces (shell, bash,\dots), or APIs
\item An object can be an instance of a class, if the class is the generating class of the object
\end{itemize}

\subsection{Information Hiding}
\begin{itemize}
\item For its clients, an attribute may be:
\begin{itemize}
\item Secret
\item Read-only
\item Read, but partially write restricted (only certain things are allowed to be written)
\item Writing one or more classes in curly brackets after the keyword \textsl{feature} exports these 
features only to these classes and its descendants. If no class is listed, the features 
are exported to \textsl{ANY}. 
\end{itemize}
Information hiding only applies to use by clients using dot or infix notation. Unqualified calls are not subject to information hiding. 
\end{itemize}

\subsection{Control structures}
\todo[inline]{Requires a lot of work}
\begin{itemize}
\item Sequence or compound
\item Loop
\begin{itemize}
\item Loop invariant
\begin{itemize}
\item Satisfied \textbf{after} initialization, after the \textsl{from} clause
\item Preserved by every loop iteration executed with the exit condition not satisfied. So in the end, the loop invariant \underline{and} the exit condition hold!
\end{itemize}
\item Loop variant
\begin{itemize}
\item Non-negative (i.e. $\geq$ 0) integer expression, right after initialization
\item Decreases while remaining non-negative for every iteration of the body with exit condition not satisfied. 
\end{itemize}
\end{itemize}
\item Conditional
\end{itemize}

\subsection{Contracts}
\begin{itemize}
\item Contracts are made of assertions, each containing an assertion tag and a condition (a Boolean expression)
\item Precondition
\begin{itemize}
\item Property that a feature imposes on every client
\item If there is no \textsl{require} clause, is treated as one, with one only being true.
\end{itemize}
\item Postcondition
\begin{itemize}
\item Property that a feature guarantees every client 
\item Can make use of keyword \textsl{old}\todo{Check for an explanation of this keyword}
\end{itemize}
\item Class invariant
\begin{itemize}
\item The invariant expresses consistency requirements between queries of a class
\end{itemize}
\end{itemize}

\subsection{Miscellaneous}
\begin{itemize}
\item Semistrict operators
\begin{itemize}
\item Let us define the order of expression evaluation
\item \textbf{and then} is the semistrict version of \textbf{and}. Use it if a condition only makes sense when another is true.
\item \textbf{or else} is the semistrict version of \textbf{or}. Use it if a condition only makes sense when another is false
\item\textbf{implies} is always semstrict!
\end{itemize}
\end{itemize}

\section{Describing syntax}
\subsection{BNF}
Backus-Naur-Form (BNF): A metasyntax used to express context-free grammars. A formal way to describe formal languages. It consists of the following parts
\begin{itemize}
\item \textbf{Delimiters}: Fixed tokens of the languages vocabulary, such as keywords and special symbols
\item \textbf{Constructs}: They represent structures of the language, for instance \emph{Conditional}. A particular instance of a construct os known as a specimen of the construct. There are two kinds of constructs:
\begin{itemize}
\item \textbf{Nonterminal construct}: They are defined by a production
\item \textbf{Terminal construct}: Terminal constructs such as \emph{Identifier} or \emph{Integer} are not defined by this grammar, they are described at the lexical level
\end{itemize}
\item \textbf{Productions}: They are associated with a particular construct and specify their specimens
\end{itemize}
Each production defines the syntax of specimens of a particular construct, in terms of other constructs and delimiters. An example for a production (not BNF-E) 
\[A=B|C[D]\{ E``;''\}^*\]
Depending on the right side of a production, they can be separated into three kinds:
\begin{itemize}
\item \textbf{Concatenation}: This production lists zero or more constructs, some may enclosed in brackets and said to be \textbf{optional}
\item\textbf{Choice}: Listing one or more constructs, separated by vertical bars. A choice specifies that every specimen of the construct on the left consists of exactly one specimen of one of the constructs on the right
\item \textbf{Repetition}: A construct, enclosed in curly brackets, followed by a star. This indicates zero or more occurrences of the construct, Example: $A=\{ B\}^*$. The star might be replaced by a plus, indicating one or more repetition
\end{itemize}
\subsection{BNF-E}
\begin{itemize}
\item Every non-terminal must appear on the left side of exactly one production, called its defining production. 
\item Every production must be of one kind: either concatenation, choice or repetition
\item There is also a major change in the repetition production. Instead of \[A=[B \{\text{terminal }B\}^*]\] one may write \[A=\{B\text{ terminal }\dots\}^*\] The same is also true for the plus instead of a star.
\end{itemize}

\subsection{Regular Grammar}
The regular grammar is generally used to describe the terminal construct, which could be done using BNF, but can be achieved more easily using a regular grammar. The rules are quite similar, although different:
\begin{itemize}
\item The use of \textbf{choice} is no problem, possibly with character intervals
\item There are also \textbf{concatenations}, although they do not assume breaks (spaces, new lines, \dots) between elements. But you may define them explicitly using a lexical construct.
\item \textbf{Repetitions} have a different, simpler form; $A^*$ or $A^+$, following the same rules
\item No \textbf{recursion} is allowed whatsoever. As a result you may write any language in a single regular expression
\item Unlike BNF-E, you may mix different kinds of productions 
\end{itemize}

\section{Inheritance and Genericity}
\subsection{Inheritance}
\begin{itemize}
\item Terminology\todo[inline]{Check for a better explanation?}
\begin{itemize}
\item A class is a \textbf{parent} to another, if the other inherits directly from it, i.e. class $A$ is a parent to class $B$, if class $B$ inherits from class $A$
\item The \textbf{descendants} of a class are the class itself and (recursively) the descendants of its heirs (parents). \textbf{Proper descendant} excludes the class itself.
\end{itemize}
\item \textbf{Features of classes}
\begin{itemize}
\item They can be \textbf{inherited} if it is a feature of one of the parents of the class. They can also be \textbf{immediate} if it is declared in the class. In this case, the class is said to introduce the feature
\item Fully implemented features are called \textbf{effective}, otherwise one may call them \textbf{deferred}
\end{itemize}
\item Contracts 
\begin{itemize}
\item The \textbf{invariant} of a class automatically includes the invariant clause from all its parents, and-ed
\item If no pre-/post condition is explicitly stated, features inherit the contracts from their parents
\item One can weaken the precondition with the keyword \inline{require else}, resulting in a precondition \inline{orig_pre} and \inline{new_pre}
\item One can strengthen the postcondition with the keyword \inline{ensure then}, resulting in a postcondition \inline{orig_post} and \inline{new_post}
\end{itemize}
\end{itemize}

\subsubsection{Multiple Inheritance}
\begin{itemize}
\item \textbf{Name clash}: If \inline{C} inherits both from \inline{A} and \inline{B}, which both have a feature \inline{f}, then we have a name clash. To resolve it, we can redefine one feature as \inline{A_f}. A name clash must be resolved, unless it is:
\begin{itemize}
\item Under repeated inheritance (the feature of \inline{f} in \inline{A} and \inline{B} comes from a common ancestor, for instance \inline{ANY})
\item If at most one of the features \inline{f} is effective, and all others are deferred. In that case (only one feature is effective), the features are said to be \textbf{merged}. Merging also works when one or more features are renamed. The merging happens after renaming!
\item If more than one feature are effective, merging can still help. We can \textbf{undefine} effective features, so that they are deferred again. Syntax: \inline{undefine a,b,c end}. It is even possible to fist rename a feature, undefine it and then merge it. 
\end{itemize}
\item Repeated Inheritance
\begin{itemize}
\item Features, not renamed along any of the inheritance paths, will be shared
\item Features, inherited under different names will be replicated
\item A potential ambiguity arises because of polymorphism and dynamic binding when class \inline{C} inherits from \inline{B} and \inline{A}, but \inline{A} redefines now copy (a feature of the common ancestor \inline{ANY}). In this case, a simple rename will not be enough, we have to select (\inline{select copy, f end}) the features from one parent, and rename the ones from the other.
\end{itemize}
\end{itemize}

\subsection{Genericity}
\begin{itemize}
\item Terminology:
\begin{itemize}
\item A \textbf{formal generic paameter} is the parameter in the class, e.g. \inline{LIST[G]} with \inline{G} as formal generic parameter.
\item An \textbf{actual generic parameter} is the actual type passed as parameter in a type, e.g. \inline{LIST[INTEGER]} with \inline{INTEGER} being the actual generic parameter. 
\item One can obtain a \textbf{generic derivation} of a generic class by passing a type
\end{itemize}
\item Types
\begin{itemize}
\item \textbf{Unconstrained} genericity: Any generic type is allowed. Example: \inline{LIST[G]}, which is the same as \inline{LIST[G->ANY]}
\item \textbf{Constrained} genericity: Only descendants are allowed as generic type. Example: \inline{LIST[G->NUMERIC]}
\end{itemize}
\end{itemize}

\subsection{Static Typing}
\begin{itemize}
\item \textbf{Type-safe call} (during execution). A feature call \inline{x.f} such that the object attached to \inline{x} has a feature corresponding to \inline{f}
\item \textbf{Static type checker} is a program-processing tool (such as a compiler) that guarantees for any program that it accepts, that any call in any execution will be type-safe
\item A programming language is called \textbf{statically typed language} if it is possible to write a static type checker
\end{itemize}

\subsection{Polymorphism}
\begin{itemize}
\item \textbf{Polymorphism} is the existence of the following possibilities:
\begin{itemize}
\item An \textbf{attachment} (assignment or argument passing) is \textbf{polymorphic} if its target variable and source expression have different types.
\item An entity or expression is polymorphic if it may at runtime, as a result of polymorphic attachments, become attached to objects of different types. 
\item A \textbf{container data structure} is polymorphic if it may contain references to objects of different types.  
\end{itemize}
\item The \textbf{static type} of an entity is the type used in its declaration in the class text. Similarly, the \textbf{dynamic type} of an entity is the type of the object, it is attached to. The type system ensures that the dynamic type of an entity will always conform to its static type.
\item \textbf{Conformance}
\begin{itemize}
\item A reference type \inline{U} \textbf{conforms} to a reference type \inline{T} if either:
\begin{itemize}
\item They have no generic parameters, and \inline{U} is a descendant of \inline{T}
\item They both are generic derivations with the same number of actual generic parameters, the base class of \inline{U} is a descendant of the base class of \inline{T}, and every actual parameter of \inline{U} (recursively) conforms to the corresponding actual parameter of \inline{T}
\end{itemize}
\item An expanded type only conforms to itself
\end{itemize}
\item \textbf{Object test}. Test the dynamic type of an object, e.g. \inline{if \{r:TYPE\} obj then ... end}
\begin{itemize}
\item \inline{ \{r:TYPE\} } is the object-test local, and only available in the \inline{then}-part, not in the \inline{else}-clause.
\item \inline{obj} is the object to be tested
\end{itemize}
\item \textbf{Assignment attempt}. Earlier mechanism for the object test. \inline{a?=b} assigns \inline{b} to \inline{a} if and only if \inline{b} is attached to an object whose type conforms to the type of \inline{a}. Otherwise, \inline{a} will be void
\end{itemize}

\subsection{Dynamic Binding}
\begin{itemize}
\item \textbf{Dynamic bonding} as a semantic rule is the property that any execution of a feature call will use the vesion of the feature best adapted to the type of the target object.
\end{itemize}

\section{Recursion}
\begin{itemize}
\item \textbf{Definition}: A definition is recursive if it involves one or more instances of the concept itself. Recursion is the use of a recursive definition. 
\item Recursion can be either direct (routine \inline{r} calls \inline{r}) or indirect (routine \inline{r_1} calls \inline{r_2} .. calls \inline{r_n} calls \inline{r_1})
\item To be useful, a recursive definition should ensure that:
\begin{enumerate}[R1:]
\item There is at least one non-recursive branch
\item Every recursive branch occurs in a context that differs from the original
\item For every recursive branch, the change of context R2 brings it closer to at least one of the non-recursive cases R1
\end{enumerate}
\item Recursive calls cause (without further optimization) a run-time penalty: the stack of preserved calls needs to be maintained. Various optimizations are possible:
\begin{itemize}
\item Recursive schemes can sometimes be replaced by a loop (\textbf{recursive elimination})
\item \textbf{Tail recursion} (last instruction of a routine is a recursive call) can usually be eliminated 
\end{itemize}
\item \textbf{Recursive variant}: Every recursive routine should use a recursion variant, an integer quantity associated with any call, such that
\begin{itemize}
\item The variant is always $\geq 0$
\item If a routine execution starts with variant value $v$, the value $v'$ for any call satisfies $0\leq v'\leq v$
\end{itemize}
\end{itemize}

\section{Data Structures}
\subsection{Trees}
\subsubsection{Binary Trees}
\begin{itemize}
\item A binary tree \inline{G}, for an arbitrary data type \inline{G}, is a finite set of items called nodes, each containing a value of type \inline{G}, such that the nodes, if any, are divided into three disjoint parts:
\begin{itemize}
\item A single node, called the root of the binary tree
\item (Recursively) two binary trees over \inline{G}, called the left and right sub-tree
\end{itemize}
\item \textbf{Theorem:} For any node of a binary tree, there is a single downward path connecting the root to the node through successive applications of \inline{left} and \inline{right} links.
\item \textbf{Traversals}
\begin{itemize}
\item \textbf{In-order}: traverse left sub-tree, visit root, traverse right sub-tree
\item \textbf{Pre-order}: visit root, traverse left sub-tree, traverse right sub-tree
\item\textbf{Post-order}: traverse left sub-tree, traverse right sub-tree, visit root
\end{itemize}
\item \textbf{Binary search tree}: This is a tree over a sorted set \inline{G} if for every node \inline{n}:
\begin{itemize}
\item For every node \inline{x} of the left sub-tree of \inline{n}: \inline{x.item} $\leq$ \inline{n.item}
\item For every node \inline{x} of the right sub-tree of \inline{n}: \inline{x.item} $\geq$ \inline{n.item}
\end{itemize}
\item In a binary search tree, average behavior for insertion, deletion and search is $O\left( \log(n)\right)$, only worst case is $O(n)$
\end{itemize}
\todo[inline]{Add binary search tree image}

\subsection{Container data structures}
\begin{itemize}
\item Containers contain other objects, and store them in numerous ways. They differ in various properties, like the available operations, the speed of these operations and storage requirements. Some fundamental operations on a container:\\
\begin{center}
\begin{tabular}{|l|l|}
\hline
Insertion & Add an item\\\hline
Removal & Remove an occurence (if any) of an item\\\hline
Wipeout & Remove remove all occurences of an item\\\hline
Search & Find out if a given item is present \\\hline
Iteration (or trasversal) & Apply a given operation to every item\\\hline
\end{tabular}
\end{center}
\item The \inline{EiffelBase} classes use standard names for basic operations:
\begin{itemize}
\item Queries:
\begin{itemize}
\item \inline{is_empty: BOOLEAN}
\item\inline{has(v:G): BOOLEAN}
\item \inline{count: INTEGER}
\item \inline{item: G}
\end{itemize}
\item Commands
\begin{itemize}
\item \inline{make}
\item \inline{put(v:G)}
\item\inline{remove(v:G)}
\item\inline{wipe_out}
\item\inline{start, finish}
\item\inline{forth, back}
\end{itemize}
\end{itemize}
\item The \textbf{cursor} is present in many containers. It ranges from $0$ to $count+1$, and \inline{before} and \inline{after} hold if the cursor is not on an item. In an empty list, the cursor is at position $0$
\item \textbf{Alias notation}. A feature may be declared as follows: \inline{item(i: INTEGER) alias "[]": G assign put}. It is then possible to do \inline{a[i]} for \inline{a.item(i)} and \inline{a.item(i) := x} or \inline{a[i] := x} for \inline{a.put(x,i)}
\end{itemize}
\subsubsection{Lists}
\begin{itemize}
\item A list is a sequence of elements of a certain type. List is a general concept and has various implementations, including \inline{LINKED_LIST}, \inline{TWO_WAY_LIST}, \inline{ARRAYED_LIST}, etc.
\item Lists have a \textbf{cursor}. The current cursor position can be obtained by the query \inline{index}. The element at this position is generally obtained by \inline{item} and there are many other queries about the cursor position: \inline{after}, \inline{before}, \inline{off}, \inline{is_first}, \inline{is_last}
\item There are also various commands for \textbf{cursor movement}: \inline{start}, \inline{finish}, \inline{forth}, \inline{back}, \inline{go_i_th}
\item \textbf{Adding and removing} elements is done using: \inline{put_front}, \inline{put_left}, \inline{put_right}, \inline{extend}, \inline{remove} (item at cursor position)
\end{itemize}
\subsubsection{Arrays}

\end{document}