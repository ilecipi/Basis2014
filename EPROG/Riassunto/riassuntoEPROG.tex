\documentclass[a4paper]{article}
\usepackage{todonotes}


\begin{document}

\section{Introduction}
\subsection{Basic definitions}
\begin{itemize}
\item Feature
\begin{itemize}
\item Command: No return value, but does modify objects. On the synctactical level, it is an instruction
\item Query: Returns a value, but does not modify any objects. The syntax equivalent is the expression.
\begin{itemize}
\item Functions get their results through computation
\item Attributes are values directly stored in memory
\end{itemize}
For queries, there is the uniform access principle, which states that it doesn't matter to the client whether a query is implemented as a function or attribute. Features should be accessible to clients the same way whether implemented by storage or by computation.
\item Creation Procedure: Commands to initiate objects, can be several. There is also a \textsl{default\_create}, which is inherited by all classes, and does nothing by default. 
\end{itemize}
\item Feature Calls
\begin{itemize}
\item Unqualified calls: Feature calls which apply to the current object
\item Qualified calls: Feature calls which apply to a certain object, causing this object to become the current object. 
\end{itemize}
\item Class clauses
\begin{itemize}
\item Indexing
\item Inheritance
\item Creation
\item Feature
\item Invariant
\end{itemize}
\item Specimen: A syntactic element, such as a class name or an instruction, but no delimiters. The type of a specimen is its construct. See Describing syntax\todo{Add reference}
\item Abstract syntax tree: Shows the syntax structure with all its specimens, but obviously without any delimiters, A tree has nodes, each one of the following kind:
\begin{itemize}
\item Root: Node with no incoming branch.
\item Leaf: Node without outgoing branches
\item Internal node: Neither of the former
\end{itemize}
\item Basic elements of a program text:
\begin{itemize}
\item Terminals
\begin{itemize}
\item Identifiers: Names chosen by the programmer
\item Keywords
\item Special symbols, such as a period
\end{itemize}
\end{itemize}
\item Describing a program
\begin{itemize}
\item Semantic rules: Define the effect of programming, satisfying the syntax rules
\item Syntax rules: Define how to make up specimens out of tokens satisfying the lexical rules
\item Lexical rules: Define how to make up tokens out of characters
\end{itemize}
\item Syntax: The way you write a program; characters grouped into words, grouped into bigger structures. 
\item Semantics: The effect you expect from this program at runtime
\item Identifier: Name chosen by the programmer to represent certain program elements, such as classes, features or runtime values. If it denotes a runtime value, it is called an identity or variable if it can change its value. During execution, an entity may become attached to an object. 
\item Executing a system consists of creating a root object, which in an instance of a designated class from the system, the root class, using a designated creation procedure, called its root procedure.
\end{itemize}
\subsection{Variables}
\begin{itemize}
\item Types
\begin{itemize}
\item Reference types: Entities with a reference value
\item Expanded types: Entities with an object as a value
\item A type is one of:
\begin{itemize}
\item A non-generic class
\item A generic derivation, i.e. the name of a class followed by a list of types, the actual generic parameters, in brackets
\end{itemize}
\end{itemize}
\item Setters: It is possible to make assignments such as $x.att:=val$, which is shorthand for $x.set\_att(val)$
\item Effect of an assignment
\begin{itemize}
\item Reference types: Reference assignment
\item Expanded types: Value copy
\end{itemize}
\item Variable copy
\begin{itemize}
\item Shallow object duplication (creates a new object): $b:=a.twin$
\item Deep object duplication (creates a new object): $b:=a.deep\_twin$
\item Shallow field-by-field copy (does not create an object): $b.copy(a)$
\end{itemize}
\end{itemize}
\todo[inline]{Maybe add the objet creation diagram??}

\subsection{Interface}
\begin{itemize}
\item A client of a software mechanism is a system of any kind - such as a software element or a human - that uses it. For its client, the mechanism is a supplier
\item Interface: The description of techniques enabling clients to use these mechanisms. For example: GUIs (Graphical User Interface), command line interfaces (shell, bash,\dots), or APIs
\item An object can be an instance of a class, if the class is the generating class of the object
\end{itemize}

\subsection{Information Hiding}
\begin{itemize}
\item For its clients, an attribute may be:
\begin{itemize}
\item Secret
\item Read-only
\item Read, but partially write restricted (only certain things are allowed to be written)
\item Writing one or more classes in curly brackets after the keyword \textsl{feature} exports these 
features only to these classes and its descendants. If no class is listed, the features 
are exported to \textsl{ANY}. 
\end{itemize}
Information hiding only applies to use by clients using dot or infix notation. Unqualified calls are not subject to information hiding. 
\end{itemize}

\subsection{Control structures}
\todo[inline]{Requires a lot of work}
\begin{itemize}
\item Sequence or compound
\item Loop
\begin{itemize}
\item Loop invariant
\begin{itemize}
\item Satisfied \textbf{after} initialization, after the \textsl{from} clause
\item Preserved by every loop iteration executed with the exit condition not satisfied. So in the end, the loop invariant \underline{and} the exit condition hold!
\end{itemize}
\item Loop variant
\begin{itemize}
\item Non-negative (i.e. $\geq$ 0) integer expression, right after initialization
\item Decreases while remaining non-negative for every iteration of the body with exit condition not satisfied. 
\end{itemize}
\end{itemize}
\item Conditional
\end{itemize}

\subsection{Contracts}
\begin{itemize}
\item Contracts are made of assertions, each containing an assertion tag and a condition (a Boolean expression)
\item Precondition
\begin{itemize}
\item Property that a feature imposes on every client
\item If there is no \textsl{require} clause, is treated as one, with one only being true.
\end{itemize}
\item Postcondition
\begin{itemize}
\item Property that a feature guarantees every client 
\item Can make use of keyword \textsl{old}\todo{Check for an explanation of this keyword}
\end{itemize}
\item Class invariant
\begin{itemize}
\item The invariant expresses consistency requirements between queries of a class
\end{itemize}
\end{itemize}

\subsection{Miscellaneous}
\begin{itemize}
\item Semistrict operators
\begin{itemize}
\item Let us define the order of expression evaluation
\item \textbf{and then} is the semistrict version of \textbf{and}. Use it if a condition only makes sense when another is true.
\item \textbf{or else} is the semistrict version of \textbf{or}. Use it if a condition only makes sense when another is false
\item\textbf{implies} is always semstrict!
\end{itemize}
\end{itemize}

\section{Describing syntax}





\end{document}