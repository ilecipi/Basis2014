%% Memoir layout setup

%% NOTE: You are strongly advised not to change any of them unless you
%% know what you are doing.  These settings strongly interact in the
%% final look of the document.

% Dependencies
\usepackage{array}
\usepackage{wrapfig}
\usepackage{multicol}
\usepackage[thmmarks]{ntheorem}

%Added for formula sheet
\usepackage{framed,color}

\usepackage{enumitem}
\setlist{nolistsep,leftmargin=*}
\setitemize{itemsep=0pt}


\columnsep 30pt
\columnseprule \normalrulethickness

%Length of separation between items in lists
%\usepackage{enumitem}
%\setlist{nosep}

%Define settings for the paper layout, like landscape and margins
\usepackage[
	a4paper,
	landscape,
	left=.75cm,
	right=.75cm,
	top=.75cm,
	bottom=0.5cm,
	includeheadfoot
]{geometry}

%Still to investigate!! %%%%%%%%%%%%%%%%%%%%%%%%%%%%%%%%%%%%%%
\renewcommand{\arraystretch}{1}
\setlength{\parskip}{1ex plus 0.5ex minus 0.2ex}



% Define the default sans serif font as the lighter computer modern bright by
% D. Knuth.
\renewcommand{\sfdefault}{cmbr}

\makeatletter

% Set the way pages are layed out (headers and page numbering)
\makepagestyle{mystyle}
\makeheadrule{mystyle}{\textwidth}{\normalrulethickness}
\makeoddhead{mystyle}{\textsc{\@title}}{}{\emph{\@author} -- \today{}}
\makeevenhead{mystyle}{\textsc{\@title}}{}{\emph{\@author} -- \today{}}
\makeoddfoot{mystyle}{}{\small -- \thepage{} --}{}
\makeevenfoot{mystyle}{}{\small -- \thepage{} --}{}
\pagestyle{mystyle}

% Use the newly defined style
\chapterstyle{VZ14}



%Redefine chapter command to include a black box around it
\renewcommand{\chapter}[1]{
\definecolor{shadecolor}{gray}{0.1}
%\vspace{-5mm}%NOT TRULY NEEDED
\begin{shaded}
\textbf{\centerline{\color{white}{#1}}}
\end{shaded}
\vspace{-3mm}
}
\makeatother

%Redefine chapter command to include a shaded box around it
\renewcommand{\section}[1]{
\definecolor{shadecolor}{gray}{0.8}
%\vspace{-3mm}
\begin{shaded}
\vspace{-1mm}
{\centerline{\textbf{#1}}}
\vspace{-1mm}
\end{shaded}
\vspace{-3mm}
}
\makeatother

%Redefine chapter command to include a box around it
\renewcommand{\subsection}[1]{
\begin{framed}
\vspace{-2mm}
{\centerline{\textbf{#1}}}
\vspace{-1.5mm}
\end{framed}
\vspace{-3mm}
}
\makeatother


\makeatletter
\renewcommand{\subsubsection}{\@startsection{subsubsection}{1}{0mm}%
{-2\baselineskip}{0.8\baselineskip}%
{\hrule depth 0.2pt width\columnwidth\vspace*{1.2em}\normalsize\bfseries\sffamily}}

\renewcommand*{\thesection}{\arabic{section}}

\setparaheadstyle{\normalsize\bfseries\sffamily}
\setsubparaheadstyle{\normalsize\bfseries\sffamily}

% Set captions to a more separated style for clearness
\captionnamefont{\sffamily\bfseries\footnotesize}
\captiontitlefont{\sffamily\footnotesize}
\setlength{\intextsep}{16pt}
\setlength{\belowcaptionskip}{1pt}

% Set section and TOC numbering depth to subsection
\setsecnumdepth{subsection}
\settocdepth{subsection}

% Turn off american style paragraph indentation and add some space to be
% printed when a new paragraph starts.

\setlength{\parindent}{0pt}
\addtolength{\parskip}{1pt}

\setlength{\droptitle}{-2em}
\renewcommand{\maketitlehookb}{\vspace{-1em}}
\date{}

% A bit spacier tabulars and lists
\setlength{\extrarowheight}{4pt}
\setlength{\itemsep}{10pt}
% \renewcommand{\arraystretch}{1.2}

% % This provides a frontend to set the lecture date into the header
% \newcommand{\lecturedate}[1]{\def\@lecdate{#1}}
% \makeoddhead{Ruled}{\@lecdate}{}{\normalfont\rightmark}

\makeatother

% This defines how theorems should look. Best leave as is.
\theoremstyle{plain}
\theoremseparator{:\quad}
\theoremprework{}
\theoremindent2em
\theoremheaderfont{\sffamily\bfseries}
\theorembodyfont{\normalfont}
\theoremsymbol{}

% redefine the over/underbrace to get more lightweight braces (borrowed from MnSymbols)
\makeatletter
\DeclareSymbolFont{largesymbolsA}{U}{txexa}{m}{n}
\def\re@DeclareMathSymbol#1#2#3#4{%
    \let#1=\undefined
    \DeclareMathSymbol{#1}{#2}{#3}{#4}}
\re@DeclareMathSymbol{\br@cext}{\mathord}{largesymbolsA}{"20}
\DeclareSymbolFont{extsymbols}{OMX}{txex}{m}{n}
\re@DeclareMathSymbol{\braceld}{\mathord}{extsymbols}{"7A}
\re@DeclareMathSymbol{\bracerd}{\mathord}{extsymbols}{"7B}
\re@DeclareMathSymbol{\bracelu}{\mathord}{extsymbols}{"7C}
\re@DeclareMathSymbol{\braceru}{\mathord}{extsymbols}{"7D}
\def\downbracefill{$\m@th%
   \braceld\mkern-1mu\cleaders\hbox{$\mkern-.5mu\br@cext\mkern-.5mu$}%
   \hfill\mkern-1mu%
   \braceru\bracelu%
   \mkern-1mu\cleaders\hbox{$\mkern-.5mu\br@cext\mkern-.5mu$}%
   \hfill\mkern-1mu\bracerd$}
 \def\upbracefill{$\m@th%
   \bracelu\mkern-1mu\cleaders\hbox{$\mkern-.5mu\br@cext\mkern-.5mu$}%
   \hfill\mkern-1mu%
   \bracerd\braceld%
   \mkern-1mu\cleaders\hbox{$\mkern-.5mu\br@cext\mkern-.5mu$}%
   \hfill\mkern-1mu\braceru$}
\makeatother

% index for multicols
\makeatletter
\renewenvironment{theindex}
  {\if@twocolumn
      \@restonecolfalse
   \else
      \@restonecoltrue
   \fi
   \begin{multicols*}{3}[\paragraph*{\indexname}]
   \markboth{\MakeUppercase\indexname}%
            {\MakeUppercase\indexname}%
   \thispagestyle{mystyle}
   \setlength{\parindent}{0pt}
   \setlength{\parskip}{0pt plus 0.3pt}
   \relax
   \let\item\@idxitem}%
  {\end{multicols*}\if@restonecol\onecolumn\else\clearpage\fi}
\makeatother